\documentclass[a4paper,11pt,]{report}
\usepackage[utf8]{inputenc}
\usepackage[portuguese]{babel}
\usepackage[T1]{fontenc}
\title{\textbf{\Huge{Placas Gráficas em Portáteis}}}
\author{Afonso Sampaio e Maria Beatriz Rocha}
\date\today
\begin{document}

\maketitle

\tableofcontents

\part{Introdução às Placas Gráficas de Portáteis}

\chapter{Definição e História}

\section{Definição e função das placas gráficas em portáteis}
As placas gráficas, também conhecidas como GPUs (Unidades de Processamento Gráfico) ou placas de vídeo, desempenham um papel fundamental em portáteis, desktops e outros dispositivos de computação quando se trata de processamento de gráficos e de vídeos.
\subsection{Definição}
Uma GPU é um componente de hardware dedicado, projetado especificamente para lidar com o processamento de gráficos e de vídeos em um computador. 
\subsection{Funções}
\subsubsection{Renderização de Gráficos}
A função principal de uma GPU de um portátil é renderizar\footnote{Obter o produto final de um processamento digital} gráficos. Isso inclui a exibição de elementos visuais na tela, como imagens, texto, vídeos e animações. A GPU alivia a carga da CPU (Unidade Central de Processamento) ao processar esses gráficos, melhorando o desempenho geral do sistema.
\pagebreak
\subsubsection{Gráficos 3D e Jogos}
São essenciais em portáteis que sejam usados em jogos ou tarefas que envolvem gráficos 3D, como modelagem 3D ou renderização de vídeos, uma vez que é responsável pelo processamento de cálculos complexos necessários para renderizar gráficos 3D em tempo real, proporcionando uma experiência de jogo suave e uma renderização rápida de projetos 3D.
\subsubsection{Aceleração de Vídeo}
Oferecem aceleração de vídeo, o que significa que elas podem decodificar e codificar vídeos de maneira mais eficiente do que a CPU. É importante para a vizualização de vídeos em altas resoluções e para edição de vídeos, visto que ajuda a economizar tempo e energia e melhora a qualidade da reprodução.
\subsubsection{Processamento Paralelo}
São, também, altamente eficientes em realizar cálculos em paralelo, o que as torna ideais para tarefas que exigem grande capacidade de processamento, como Machine Learning\footnote{Dá aos computadores a capacidade de identificar padrões em dados massivos e fazer previsões (análise preditiva)}, criptomining\footnote{Usar recursos de um sistema para resolver grandes cálculos matemáticos que resultam em alguma quantidade de criptomoeda sendo concedida aos solucionadores.}, entre outros. Muitos portáteis modernos utilizam GPUs dedicadas para acelerar estas tarefas.

\section{Evolução histórica das placas gráficas para portáteis}
\subsection{Início dos portáteis (Década de 1980)}
Nos primórdios dos portáteis, as placas gráficas eram geralmente integradas na motherboard e ofereciam um desempenho muito limitado. A maioria usava telas monocromáticas ou gráficos básicos e não havia muita capacidade gráfica dedicada.
\subsection{Década de 1990}
À medida que os portáteis se foram tornando mais populares, houve uma demanda crescente por melhores capacidades gráficas. A NVIDIA e a ATI (mais tarde adquirida pela AMD) começaram a produzir GPUs dedicadas, para portáteis. No entanto, ainda eram relativamente fracas em comparação com as versões de desktop.
\subsection{Década de 2000}
Na década de 2000 houve um aumento significativo no poder gráfico dos portáteis. As GPUs dedicadas tornaram-se mais poderosas e capazes de executar jogos e aplicações gráficas exigentes. Tecnologias como o SLI (Scalable Link Interface) da NVIDIA e o CrossFire da ATI permitiam aos portáteis serem equipados com múltiplas GPUs.
\subsection{Década de 2010}
Nesta década, houve uma revolução na indústria dos portáteis devido ao aumento do interesse em jogos de computadores. Fabricantes como a NVIDIA e a AMD começaram a lançar GPUs de alto desempenho especificamente projetadas para portáteis gaming. Estes tornaram-se populares, com telas de alta taxa de atualização e resoluções mais altas.
\subsection{Década Atual}
Hoje em dia a evolução de GPUs em portáteis continua. A NVIDIA recentemente lançou a sua última geração de GPUs, as RTX 40 para portáteis, que são baseadas na arquitetura Ada Lovelace e que oferecem um desmpenho bastante competitivo superando os seus rivais da AMD com as suas GPUs Radeon RDNA3.

\part{GPUs}

\chapter{Arquitetura de uma GPU}
A arquitetura de uma GPU refere-se ao design e à organização interna das unidades de processamento gráfico em uma GPU. Esta arquitetura determina como esta vai processa dados gráficos, realizar cálculos e exibir imagens na tela do computador. Ao longo dos anos, várias arquiteturas de placas gráficas foram desenvolvidas por diferentes fabricantes, com avanços significativos em termos de desempenho e eficiência.

\section{Primórdios de GPUs}
As GPUs passaram por uma evolução desde a Arquitetura de Barramento Fixo, que era baseada em funções fixas sem programação; passando pela Arquitetura de Transformação e Iluminação(T\&L) que consistia em separar esta duas operações do processamento dos pixeis, fornecendo assim um melhor desempenho; até chegarmos à Arquitetura de Shader Unificado, que permitiu a execução de tarefas programáveis, incluindo o uso de sombreadores para efeitos visuais mais avançados.
\pagebreak
\section{GPUs Atuais}
\subsection{Arquiteturas da NVIDIA}
A NVIDIA lançou várias arquiteturas de GPUs, mas iremos apenas citar as mais conhecidas e as que iremos utilizar nos tested de desempenho:
\begin{itemize}
 \item[•] \textbf{NVIDIA Maxwell (2014)}: A grande mudança começou aqui. As GPUS tornaram-se mais eficientes; houve melhorias de shaders\footnote{Programa de computador usado para fazer shading: a produção de níveis de cor apropriadas para uma imagem e produzir efeitos especiais ou pós-processamento de vídeo.}; foi a primeira a ter suporte total para DirectX\footnote{Conjunto de APIs que permite aos softwares dar instruções diretas para os componentes de hardware de áudio e de vídeo, melhorando assim o desempenho das aplicações na execução de recursos multimédia} 12; premitia o controlo total sobre a frequência e voltagem da GPU; o aparecimento do HEVC(H.265)\footnote{Formato de vídeo com alta eficiência de compressão para vídeos de alta resolução}, o que resultou em melhor qualidade de vídeo e menor uso da CPU; a primeira a ter suporte para G-SYNC\footnote{Tecnologia da Nvidia que elimina a "quebra" de quadros em jogos, presentes nos monitores gamer}, eliminando assim o tearing\footnote{Fenómeno que cria uma imagem destorcida}.
 \item[•] \textbf{NVIDIA Pascal (2016)}: É a arquitetura mais comum de todas, visto que possui uma boa eficiência energética juntamente com um desempenho sólido. Apesar de não ser das  arquiteturas mais recentes, continua a ter suporte para quase todas as tecnologias modernas lançadas hoje em dia, para além disso, podem ser compradas a preços acessíveis.
 \item[•] \textbf{NVIDIA Turing (2018)}: Usada nas graficas da série RTX20 onde se deu uma grande evolução: foram incluídos núcleos de Ray Tracing\footnote{Técnica avançada de renderização gráfica usada para simular o comportamento realista da luz em ambientes virtuais, como jogos, animações, filmes e aplicações de modelagem 3D.}, bem como núcleos de inteligência artificial; introdução da tecnologia DLSS\footnote{Tecnologia desenvolvida pela NVIDIA para melhorar o desempenho gráfico em jogos e aplicativos, ao mesmo tempo em que mantém ou até mesmo melhora a qualidade da imagem, utilizando a inteligência artificial para atingir estes mesmos objetivos.}(Deep Learning Super Sampling); uma maior perfomance e eficiência global bem como um aumento muito significativo do desempenho em RV; a primeira a ter suporte para DirectX 12 Ultimate; e a primeira a ter suporte para novas tecnologias de resolução, conseguindo suportar resoluções até 4K 144HZ.
\pagebreak
 \item[•] \textbf{NVIDIA Ampere (2020)}: Usada nas gráficas da série RTX30 e nas séries para profissionais Quadro, que tinha as segintes vantagens: um desempenho muito superior em relação à geração anterior, tendo também uma maior efiçiência energética; uma grande melhoria nos núcleos de Ray Tracing tornando este método bastante fluido e responsivo; melhoria nos núcleos de inteligência artificial; a capacidade de executar computação de alto desempenho sendo capaz de fazer a análise de dados e computação de alto desempenho; suporte para resoluções de 8K 60HZ
 \item[•] \textbf{NVIDIA Ada Lovelace (2022)}: Usada nas gráficas da série RTX40 que trouxe: grandes aumentos no desempenho e na eficiência das GPUs; melhoria dos núcleos de Ray Tracing bem como nos núcleos de inteligência artificial; suporte para tecnologia DLSS 3; suporte para o codec AV1\footnote{Codec de compressão de vídeo que oferece licenciamento sem royalties e que pode atingir um desempenho até 30\%melhor que o H.265}
\end{itemize}
\subsection{Arquiteturas da AMD}
Do outro lado da concorrência temos a AMD que ao longo dos anos também lançou arquiteturas bastante diferentes e com um desempenho escalar, apesar de andar sempre atrás dos calcanhares da NVIDIA não criando, assim, muita tecnologia nova e que as destinguisse das da NVIDIA. Aqui seguem as mais conhecidas e que têm uma rivalidade direta com as mencionadas da NVIDIA:
\begin{itemize}
 \item[•] \textbf{AMD Vega (2017)}: Uma arquitetura com características muito semelhantes às da NVIDIA Pascal tendo a vantagem de esta se encontrar a um preço mais acessível e oferecer um desempenho similar; usa a tecnologia HBM2 que tem uma largura de banda de memória muito superior ao tipo GDDR5 e GDDR5X encontrado em placas NVIDIA.
  \item[•] \textbf{AMD RDNA(2019)}: A AMD quase conseguiu igualar a NVIDIA nesta geração em GPUs a preços mais acessíveis; continuavam a ter uma largura de banda de memória maior mas em contrapartida estas ainda não possuiam núcleos de Ray Tracing nem núcleos de inteligência artificial como a NVIDIA além do que tinham uma melhor eficiência energética.
\pagebreak
  \item[•] \textbf{AMD RDNA 2 (2020)}: A segunda geração da RDNA trouxe algumas melhorias bastante consideráveis em relação à geração anterior: o aparecimento da tecnologia Infinity Cache\footnote{Tecnologia que permite que a GPU acesse dados frequentemente utilizados mais rapidamente tornando-a assim mais eficiente} que foi e continua a ser um dos grandes pontos fortes da AMD; o aparecimento de núcleos de Ray Tracing mas que infelizmente não conseguem sequer superar os da primeira geração da NVIDIA para além de que continuam a não possuir núcleos de inteligência artificial; a banda de memória continua a ser maior no lado da AMD; e continuam a ter um preço bem mais acessível do que as da NVIDIA e também são mais eficientes, no geral.
 \item[•] \textbf{AMD RDNA 3 (2022)}: A terceira geração da tecnologia RDNA veio melhorar as coisas introduzidas na geração anterior: a segunda geração da tecnologia Infinity Cache surgiu e veio aumentar ainda mais a banda da memória; melhoria dos núcleos de Ray Tracing que continua a não ser o ponto forte da AMD conseguindo apenas igualar a geração arquitetura Ampere; implementação de núcleos de AI; continua a estar a um preço muito mais acessível do que a concorrência mas esta arquitetura é também menos eficiente em relação à arquitetura Ada Lovelace; suporte a displayports 2.1 capazes de suportar resoluções 16K 60Hz.
\end{itemize}

\end{document}